%%%%%%%%%%%%%%%%%%%%%%% file template.tex %%%%%%%%%%%%%%%%%%%%%%%%%
%
% This is a template file for Web of Conferences Journal
%
% Copy it to a new file with a new name and use it as the basis
% for your article
%
%%%%%%%%%%%%%%%%%%%%%%%%%% EDP Science %%%%%%%%%%%%%%%%%%%%%%%%%%%%
%

\documentclass[twocolumn]{webofc}
%%% "twocolumn" for typesetting an article in two columns format (default one column)
%%%\documentclass{webofc}

\usepackage[varg]{txfonts}   % Web of Conferences font
%
% Put here some packages required or/and some personal commands
%
%
\begin{document}
%
\title{Building a Global HEP Software Training Community}
%
% subtitle is optional
%
%%%\subtitle{Do you have a subtitle?\\ If so, write it here}

\author{
    \firstname{First author} \lastname{First author}\inst{1,3}\fnsep\thanks{\email{Mail address for first author}} 
    \and
    \firstname{Second author} \lastname{Second author}\inst{2}\fnsep\thanks{\email{Mail address for second author if necessary}} \and
    \firstname{Third author} \lastname{Third author}\inst{3}\fnsep\thanks{\email{Mail address for last  author if necessary}}
    % etc
}

\institute{Insert the first address here
\and
           the second here
\and
           Last address
          }

\abstract{%
    The HSF/IRIS-HEP Software Training group provides software training skills to new researchers in High Energy Physics (HEP) and related communities. These skills are essential to produce high-quality and sustainable software needed to do the research. Given the thousands of users in the community, sustainability, though challenging, is the centerpiece of its approach. The training modules are open source and collaborative. Different tools and platforms, like GitHub, enable technical continuity, collaboration and nurture the sense to develop software that is reproducible and reusable. This contribution describes these efforts.
}
%
\maketitle
%
% SLIDES: https://indico.jlab.org/event/459/contributions/11538/attachments/9476/13737/Train_to_Sustain_SudhirMalik_CHEP2023.pdf
%
\section{Introduction}
\label{intro}
%
\begin{thebibliography}{}
%
% and use \bibitem to create references.
%
\bibitem{RefJ}
% Format for Journal Reference
Journal Author, Journal \textbf{Volume}, page numbers (year)
% Format for books
\bibitem{RefB}
Book Author, \textit{Book title} (Publisher, place, year) page numbers
% etc
\end{thebibliography}

\end{document}

% end of file template.tex

<div id='footer'><table width='100%'><tr><td class='right'><a href='http://fusioninventory.org/'><span class='copyright'>FusionInventory 9.1+1.0 | copyleft <img src='/glpi/plugins/fusioninventory/pics/copyleft.png'/>  2010-2016 by FusionInventory Team</span></a></td></tr></table></div>
